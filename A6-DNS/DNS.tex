\documentclass[12pt,letterpaper]{article}
\usepackage[utf8]{inputenc}
\usepackage[english]{babel}
%\usepackage{minted}
\usepackage{listings}
\usepackage{xcolor}

%For syntax highlighting
\definecolor{codegreen}{rgb}{0,0.6,0}
\definecolor{codegray}{rgb}{0.5,0.5,0.5}
\definecolor{codepurple}{rgb}{0.58,0,0.82}
\definecolor{backcolour}{rgb}{1,1,1}

%%Sets different parameters
\lstdefinestyle{mystyle}{
	backgroundcolor=\color{backcolour},   
    commentstyle=\color{codegreen},
    keywordstyle=\color{magenta},
    numberstyle=\tiny\color{codegray},
    stringstyle=\color{codepurple},
    basicstyle=\ttfamily\footnotesize,
    breakatwhitespace=false,         
    breaklines=true,                 
    captionpos=b,                    
    keepspaces=true,                 
    numbers=left,                    
    numbersep=5pt,                  
    showspaces=false,                
    showstringspaces=false,
    showtabs=false,                  
    tabsize=4
}
\lstset{style=mystyle}

\title{\textbf{Department of Computer Science and Engineering}}
\author{\textbf{S.G.Shivanirudh , 185001146, Semester V }}

\date{16 September 2020}

\begin{document}
\maketitle
\hrule
\section*{\center{UCS1511 - Networks Laboratory}}
\hrule 
\bigskip\bigskip

%Assignment name
\subsection*{\center{\textbf{Exercise 6: Domain Name Server using UDP}}}

%Objective
\subsection*{\flushleft{Objective:}}
\begin{flushleft}
Simulate the concept of \textbf{Domain Name Server} using UDP.
\end{flushleft}

%Code
\subsection*{\flushleft{Code:}}
\subsubsection*{\flushleft{DNS Table structure:}}
\begin{flushleft}
\lstinputlisting[language = C, firstline = 1, lastline = 125]{DNSTable.h}
\end{flushleft}
\subsubsection*{\flushleft{Server:}}
\begin{flushleft}
\lstinputlisting[language = C, firstline = 4, lastline = 77]{Server.c}
\end{flushleft}
\subsubsection*{\flushleft{Client:}}
\begin{flushleft}
\lstinputlisting[language = C, firstline = 4, lastline = 62]{Client.c}
\end{flushleft}

%Output
\subsection*{\flushleft{Output:}}
\subsubsection*{\flushleft{Server:}}
\begin{flushleft}
\lstinputlisting[language = C, firstline = 81, lastline = 128]{Server.c}
\end{flushleft}
\subsubsection*{\flushleft{Client 1:}}
\begin{flushleft}
\lstinputlisting[language = C, firstline = 67, lastline = 79]{Client.c}
\end{flushleft}
\subsubsection*{\flushleft{Client 2:}}
\begin{flushleft}
\lstinputlisting[language = C, firstline = 82, lastline = 92]{Client.c}
\end{flushleft}
\subsubsection*{\flushleft{Client 3:}}
\begin{flushleft}
\lstinputlisting[language = C, firstline = 97, lastline = 107]{Client.c}
\end{flushleft}
\hrule
%-----------------------------------------------------------------------------------------------------------------------
%Recursive DNS

\subsection*{\flushleft{Objective:}}
\begin{flushleft}
Simulate the concept of \textbf{Recursive Domain Name Server} using UDP.
\end{flushleft}

%Code
\subsection*{\flushleft{Code:}}
\subsubsection*{\flushleft{Server:}}
\begin{flushleft}
\lstinputlisting[language = C, firstline = 4, lastline = 122]{RecursiveServer.c}
\end{flushleft}

%Output
\subsection*{\flushleft{Output:}}
\subsubsection*{\flushleft{Server:}}
\begin{flushleft}
\lstinputlisting[language = C, firstline = 126, lastline = 209]{RecursiveServer.c}
\end{flushleft}
\subsubsection*{\flushleft{Client 1:}}
\begin{flushleft}
\lstinputlisting[language = C, firstline = 115, lastline = 126]{Client.c}
\end{flushleft}
\subsubsection*{\flushleft{Client 2:}}
\begin{flushleft}
\lstinputlisting[language = C, firstline = 129, lastline = 140]{Client.c}
\end{flushleft}
\subsubsection*{\flushleft{Client 3:}}
\begin{flushleft}
\lstinputlisting[language = C, firstline = 143, lastline = 153]{Client.c}
\end{flushleft}
\hrule
\end{document}